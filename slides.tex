\documentclass{beamer}

\usepackage[utf8]{inputenc}
\usepackage{pgfpages}
\usepackage{stmaryrd}
\usepackage{amsmath}
\usepackage{graphicx}
\usepackage{bussproofs}

\hypersetup{pdfstartview={Fit}}
\def\limp {\mathbin{{-}\mkern-3.5mu{\circ}}}



\setbeamertemplate{navigation symbols}{}
\setbeamertemplate{footline}
  {\hfill {\normalsize \insertframenumber{}/\inserttotalframenumber{}}}

  
\newcommand{\hastype}{\mathop{:}}

\newcommand{\dand}{\mathbin{\overline{\land}}}
\newcommand{\dnot}{\mathop{\overline{\lnot}}}
\newcommand{\dimpl}{\mathbin{\overline{\to}}}
\newcommand{\dexists}{\mathop{\overline{\exists}}}
\newcommand{\dforall}{\mathop{\overline{\forall}}}

\newcommand{\hsbind}{\mathbin{\texttt{>>=}}}
\newcommand{\hsrevbind}{\mathbin{\texttt{=<<}}}
\newcommand{\hsseq}{\mathbin{\texttt{>>}}}
\newcommand{\occons}{\mathbin{::}}

\newcommand{\statecps}[3]{\llbracket #3 \rrbracket^{#2}_{#1}}
\newcommand{\cps}[2]{\llbracket #2 \rrbracket^{#1}}

\newcommand{\sem}[1]{\llbracket #1 \rrbracket}
\newcommand{\intens}[1]{\overline{#1}}

\newcommand{\obj}[1]{\text{Obj}(#1)}
\newcommand{\inl}[1]{\text{inl}(#1)}
\newcommand{\inr}[1]{\text{inr}(#1)}
\newcommand{\id}[1]{\text{id}_{#1}}

\newcommand{\keyword}[1]{\texttt{#1}}
\newcommand{\effect}[1]{\textbf{#1}}
\newcommand{\semdom}[1]{\textbf{#1}}
\newcommand{\handle}[2]{\keyword{with}\ #1\ \keyword{handle}\ #2}


\newcommand{\highlight}[2]{\colorbox{#1}{$\displaystyle #2$}}


\setbeamercolor{block title}{fg=ngreen,bg=white} % Colors of the block titles
\setbeamercolor{block body}{fg=black,bg=white} % Colors of the body of blocks
\setbeamercolor{block alerted title}{fg=white,bg=dblue!70} % Colors of the highlighted block titles
\setbeamercolor{block alerted body}{fg=black,bg=dblue!10} % Colors of the body of highlighted blocks
% Many more colors are available for use in beamerthemeconfposter.sty




\usepackage[backend=biber, bibstyle=minimal, citestyle=authoryear-comp]{biblatex}
\addbibresource{effects-paper.bib}


\begin{document}

\title[Effects \& Handlers in Semantics]{Algebraic Effects and Handlers \\
  in Natural Language Interpretation}
\author{Jiří Maršík and Maxime Amblard}
\institute[LORIA, Université de Lorraine, Inria]
{
  LORIA, UMR 7503, Université de Lorraine, CNRS, Inria, Campus Scientifique, \\
  F-54506 Vand\oe uvre-lès-Nancy, France
}
\date[July 2014]{July 17, 2014}

\frame{\titlepage \setcounter{framenumber}{1}}


\begin{frame} \frametitle{Objectives}

\begin{enumerate}
\item Detailed semantics for a large-scale grammar of a natural language
  \vfill
\item Capturing the interactions of non-local (i.e. non-compositional)
  semantic phenomena
  \begin{itemize}
    \item anaphora
    \item in-situ quantification
    \item event arguments
    \item presupposition
    \item intensionalization
    \item extraction
    \item \ldots
  \end{itemize}
  \vfill
\item Multiple semantic phenomena in a single treatment without overly
  complicated types and terms
\end{enumerate}

\end{frame}


\begin{frame}
  \frametitle{In-situ quantification}
  \framesubtitle{\textcite{barker2002continuations}}

  \label{ex:quantification} Mary read every book. \\
  $\forall x. \semdom{book}(x) \to \semdom{read}(\semdom{Mary}, x)$

  \begin{align*}
  \sem{s}  &= o \\
  \sem{np} &= (\iota \to o) \to o
  \end{align*}

  \begin{align*}
  \sem{\textsc{read}} & : \sem{np} \to \sem{np} \to \sem{s} \\
  \sem{\textsc{read}} & : ((\iota \to o) \to o) \to ((\iota \to o) \to o) \to
  o \\
  \sem{\textsc{read}} & = \lambda s o. s (\lambda x. o (\lambda
  y. \textbf{read}(x, y)))
  \end{align*}
\end{frame}


\begin{frame}
  \frametitle{Anaphora}
  \framesubtitle{\textcite{de2006towards}}

  \label{ex:anaphora} Mary$_1$ read her$_1$ favorite book. \\
  $\semdom{read}(\semdom{Mary}, \semdom{favorite-book}(\semdom{Mary}))$

  \vspace{-2mm}

  \begin{align*}
  \sem{s}  &= \overline{o} \\
           &= \gamma \to (\gamma \to o) \to o \\
  \sem{np} &= (\iota \to \overline{o}) \to \overline{o} \\
           &= (\iota \to \gamma \to (\gamma \to o) \to o) \to \gamma \to (\gamma \to o) \to o
  \end{align*}

  \vspace{-3mm}

  \begin{align*}
  \sem{\textsc{read}} & : \sem{np} \to \sem{np} \to \sem{s} \\
  \sem{\textsc{read}} & : ((\iota \to \overline{o}) \to \overline{o}) \to
                          ((\iota \to \overline{o}) \to \overline{o}) \to
                          \overline{o} \\
  \sem{\textsc{read}} & : ((\iota \to \gamma \to (\gamma \to o) \to o)
                           \to \gamma \to (\gamma \to o) \to o) \\ & \to
                          ((\iota \to \gamma \to (\gamma \to o) \to o)
                           \to \gamma \to (\gamma \to o) \to o) \\ & \to
                           \gamma \to (\gamma \to o) \to o \\
  \sem{\textsc{read}} & = \lambda s o. s (\lambda x. o (\lambda y e
  \phi. \textbf{read}(x, y) \land \phi e))
  \end{align*}

\end{frame}




\begin{frame} \frametitle{Motivation}

  \vfill
\begin{itemize}
\item non-local phenomena + compositionality = generalizing meaning (often by
  abstracting over some new parameter)
  \vfill
\item more non-local phenomena $\Rightarrow$ more parameters $\Rightarrow$
  more complexity
  \vfill
\item most research focuses on single phenomena
\end{itemize}
  \vfill

\end{frame}
 


\begin{frame} \frametitle{Effects in Interpretation}
\vfill
\begin{center}
  semantic generalizations $\approx$ monads
\end{center}
\vspace{-7mm}
\begin{flushright}
  \parencite{shan2002monads}
\end{flushright}

\vfill

\begin{center}
  Montague's PTQ $\approx$ evaluation order + continuations
\end{center}
\vspace{-7mm}
\begin{flushright}
  \parencite{barker2002continuations}
\end{flushright}

\vfill

\begin{center}
  non-local phenomena $\approx$ computational effects \\
  $\Rightarrow$ elegant explanation of their interactions
\end{center}
\vspace{-7mm}
\begin{flushright}
  \parencite{shan2005linguistic, kiselyov2008call}
\end{flushright}
\vfill
\end{frame}


\begin{frame}
  \frametitle{Effects and Handlers}
  \framesubtitle{Introduction}

\begin{itemize}
\item Effectful operation: throws an exception containing the supplied
  argument and the current continuation
  \vfill
\item Handlers: capture the exceptions to implement the operations
  \begin{itemize}
  \item e.g. just by applying the continuation to some result
  \end{itemize}
  \vfill
\item Type-and-effect system: like Java's checked exceptions
\end{itemize}
\vfill
\begin{prooftree}
  \AxiomC{$(\effect{op} : A \to B) \in E$}
  \AxiomC{$\Gamma \vdash V : A$}
  \AxiomC{$\Gamma, x : B \vdash_E M : C$}
  \TrinaryInfC{$\Gamma \vdash_E \effect{op}\ V\ (\lambda x. M) : C$}
\end{prooftree}
  
\begin{flushright}
  \parencite{kammar2013handlers}
\end{flushright}
\end{frame}

\begin{frame}
\frametitle{Effects and Handlers}
\framesubtitle{Advantages}

\begin{itemize}
\item Easier to combine multiple effects in a single semantics
  \parencite{cartwright1994extensible} \parencite{kiselyov2013extensible}
\end{itemize}
\vspace{6mm}
$$
\sem{C_E} = C + \sum_{(\effect{op} : A \to B) \in E} A \times \sem{C_E}^B
$$
\vspace{2mm}
\begin{center}
\setlength{\tabcolsep}{1pt}
\begin{tabular}{r c l}
  ``effects + handlers'' & : & ``delimited continuations'' \\
  & = & \\
  ``while'' & : & ``goto''
\end{tabular}
\end{center}
\begin{flushright}
  \parencite{bauer2012programming}
\end{flushright}
\end{frame}


\begin{frame}
  \frametitle{Translating Dynamic Logic}
  \framesubtitle{Effects and Handlers}
\begin{itemize}
\item Effectful operations
  \begin{align*}
    \effect{get} &: 1 \to \gamma^{\{\effect{get}\}} \\
    \effect{fresh} &: 1 \to \iota^{\{\effect{fresh}\}} \\
    \effect{assert} &: o \to 1^{\{\effect{assert}\}} \\
    \effect{scope\_over} &: ((\iota \to o) \to o) \to \iota^{\{\effect{scope\_over}\}} \\
    \effect{move} &: 1 \to \iota^{\{\effect{move}\}}
  \end{align*}

\item Handlers
  \begin{align*}
    drs &: \gamma \to (o^{\{\effect{get}; \effect{fresh}; \effect{assert}|\rho\}} \Rightarrow o^{\rho}) \\
    tensed\_clause &: o^{\{\effect{scope\_over}|\rho\}} \Rightarrow o^{\rho} \\
    extract &: \alpha^{\{\effect{move}|\rho\}} \Rightarrow (\iota \to \alpha^{\rho})
  \end{align*}
\end{itemize}
\end{frame}



\begin{frame}
\frametitle{Translating Dynamic Logic}
\framesubtitle{Logical Connectives}
\vspace{-4mm}
\begin{align*}
  \onslide<1->{\dexists P &= \lambda e \phi. \exists x. P x (x \occons e) \phi \\}
  \onslide<2->{\dexists P &= P\ (\effect{fresh}\ ()) \\[3mm]}
  \onslide<1->{\dnot A &= \lambda e \phi. \lnot (A e (\lambda e'. \top)) \land \phi e \\}
  \onslide<2->{\dnot A &= \lnot (\handle{drs\ (\effect{get}\ ())}{A}) \\[3mm]}
  \onslide<1->{A \dand B &= \lambda e \phi. A e (\lambda e'. B e' \phi) \\}
  \onslide<2->{A \dand B &= A \land B \\[6mm]}
  \onslide<3>{A \dimpl B &= \dnot (A \dand \dnot B) \\}
  \onslide<3>{\dforall P &= \dnot \dexists x. \dnot P x}
\end{align*}
\end{frame}

\begin{frame}
\frametitle{Translating Dynamic Logic}
\framesubtitle{Lexical Items}
\vspace{-5mm}
\begin{align*}
  \onslide<1->{\sem{\textsc{she}} &= \lambda k e \phi. k (sel_{she} (e)) e \phi \\}
  \onslide<2->{\sem{\textsc{she}} &= \lambda k. k (sel_{she} (\effect{get}\ ())) \\}
  \onslide<3>{\sem{\textsc{she}} &= \{ sel_{she} (\effect{get}\ ()) \} \\[2mm]}
  \onslide<1->{\sem{\textsc{something}} &= \lambda k. \dexists x. (k\ x) = \lambda k e \phi.\ \exists x. k x (x \occons e) \phi \\}
  \onslide<2->{\sem{\textsc{something}} &= \lambda k. \dexists x. (k\ x) = \lambda k. k (\effect{fresh}\ ()) \\}
  \onslide<3>{\sem{\textsc{something}} &= \{ \effect{fresh}\ () \} \\[2mm]}
  \onslide<1->{\sem{\textsc{every}} &= \lambda n k. \dforall x. (n\ x) \dimpl (k\ x) \\}
  \onslide<2->{\sem{\textsc{every}} &= \lambda n k. \dforall x. (n\ x) \dimpl (k\ x) \\}
  \onslide<3>{\sem{\textsc{every}} &= \lambda n. \{ \effect{scope\_over}\ (\lambda k. \dforall x. (n\ x) \dimpl (k\ x)) \} \\[2mm]}
  \onslide<1->{\sem{\textsc{read}} &= \lambda S O. S (\lambda s. O (\lambda o e \phi. \semdom{read}(s, o) \land \phi e)) \\}
  \onslide<2->{\sem{\textsc{read}} &= \lambda S O. S (\lambda s. O (\lambda o. \semdom{read}(s, o))) \\}
  \onslide<3>{\sem{\textsc{read}} &= \lambda s_t o_t. \{ \handle{tensed\_clause}{\semdom{read}(s_t!, o_t!)} \}}
\end{align*}
\end{frame}

\begin{frame}
\frametitle{Translating Dynamic Logic}
\framesubtitle{Lexical Items (cont'd)}

\begin{center}
\setlength{\tabcolsep}{1pt}
\begin{tabular}{r l}
   \onslide<1->{$\sem{\textsc{some}}$ & $= \lambda n k. \dexists x. (n\ x) \dand (k\ x)$ \\[1mm]}
   \onslide<2->{$\sem{\textsc{some}}$ & $= \lambda n k. \dexists x. (n\ x) \land (k\ x)$ \\[0.5mm]
      & $= \lambda n k. k ($
     \begin{tabular}[t]{l}
       $\keyword{let}\ x = \effect{fresh}\ ()\ \keyword{in}$ \\
       $\keyword{let}\ () = \effect{assert}\ (n\ x)\ \keyword{in}$ \\
       $x\ )$
     \end{tabular} \\[1mm]}

    \onslide<3>{$\sem{\textsc{some}}$ & $=\lambda n. \{\ \effect{scope\_over}\ (\lambda k. \dexists x. (n\ x) \land (k\ x))\ \}$ \\[0.5mm]
     & $= \lambda n. \{$
     \begin{tabular}[t]{l}
       $\keyword{let}\ x = \effect{fresh}\ ()\ \keyword{in}$ \\
       $\keyword{let}\ () = \effect{assert}\ (n\ x)\ \keyword{in}$ \\
       $x\ \}$
     \end{tabular} \\}
   \\[2mm]
   \onslide<1->{$\sem{\textsc{who}}$ & $= \lambda r n x. n x \dand r (\lambda k. k x)$ \\[1mm]}
   \onslide<2->{$\sem{\textsc{who}}$ & $= \lambda r n x. n x \land r (\lambda k. k x)$ \\[1mm]}
   \onslide<3>{$\sem{\textsc{who}}$ & $= \lambda r n x. n x \land r x$}
\end{tabular}
\end{center}
\end{frame}


\begin{frame}
\frametitle{Treating Extraction as an Effect}

\begin{center}
 \setlength{\tabcolsep}{1pt}
 \begin{tabular}{r l}
  $\sem{\textsc{who}}$ & $: \sem{s}^{\{\effect{move}|\rho\}} \to \sem{n} \to \sem{n}^{\rho}$ \\[1mm]
  $\sem{\textsc{who}}$ & $= \lambda r_t n. \lambda x.$
   \begin{tabular}[t]{l}
     $\keyword{let}\ r = \handle{extract}{r_t!}\ \keyword{in}$ \\
     $(n\ x) \land (r\ x)$
   \end{tabular} \\
   \\
 & \begin{tabular}{r l}
  $r_t$ & $: o^{\{\effect{move}|\rho\}}$ \\
  $r$ & $: \iota \to o^{\rho}$ \\
  $n$ & $: \iota \to o$ \\
  $x$ & $: \iota$
   \end{tabular} \\
   \\[4mm]
  $\sem{\epsilon}$ & $: \sem{np}^{\{\effect{move}\}}$ \\[1mm]
  $\sem{\epsilon}$ & $= \{ \effect{move}\ () \} $
  \end{tabular}
\end{center}

\end{frame}


\begin{frame}
  \frametitle{Example - Syntax}
  \framesubtitle{Every farmer who owns a donkey beats it.}
  \includegraphics[width=\textwidth]{syntax2.pdf}
\end{frame}


\begin{frame}
  \frametitle{Example - Semantics}
  \framesubtitle{Every farmer who owns a donkey beats it.}
  \includegraphics[width=\textwidth]{semantics.pdf}
\end{frame}


\begin{frame} \frametitle{Conclusion}

\vfill
We have:
\vfill
\begin{itemize}
\item motivated the use of algebraic effects and handlers in semantics.
  \vfill
\item translated de Groote's continuation-based dynamic logic
  \parencite{de2006towards} to effects, reconstructing notions from DRT.
  \vfill
\item treated extraction as an effect in interpretation instead of using
  hypothetical reasoning and lambda abstractions in the syntax.
\end{itemize}
\vfill

\end{frame}


\begin{frame} \frametitle{Future Work}

\vfill
We would like to:
\vfill
\begin{itemize}
\item show how effects and handlers apply to the other non-local phenomena
  (presupposition, event arguments, optional items, intensionalization).
  \vfill
\item build a fragment that combines all of these.
  \vfill
\item design a calculus with algebraic effects and handlers and a suitable
  evaluation order (CBV vs CBN).
\end{itemize}
\vfill
  
\end{frame}

\begin{frame}
  \frametitle{Thank You}
  \vfill
  \begin{center}
    Questions?
  \end{center}
  \vfill
  \begin{center}
    Would you like to know more?

    \vspace{6mm}

    \scriptsize{\url{http://jirka.marsik.me/research/algebraic-effects-and-handlers-in-natural-language-interpretation}}
  \end{center}
\end{frame}

\begin{frame}
  \frametitle{References}
  \footnotesize{\printbibliography}
\end{frame}

\end{document}
