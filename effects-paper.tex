\documentclass[a4paper,11pt]{easychair}

\usepackage[utf8]{inputenc}
\usepackage{listings}
\usepackage{color}
\usepackage{amsmath}
\usepackage{stmaryrd}
\usepackage{verbatim}
\usepackage{graphicx}
\usepackage{natbib}
\usepackage{gb4e}
\usepackage{bussproofs}


\newcommand{\dand}{\mathbin{\overline{\land}}}
\newcommand{\dnot}{\mathop{\overline{\lnot}}}
\newcommand{\dimpl}{\mathbin{\overline{\to}}}
\newcommand{\dexists}{\mathop{\overline{\exists}}}
\newcommand{\dforall}{\mathop{\overline{\forall}}}

\newcommand{\hsbind}{\mathbin{\texttt{>>=}}}
\newcommand{\hsrevbind}{\mathbin{\texttt{=<<}}}
\newcommand{\hsseq}{\mathbin{\texttt{>>}}}
\newcommand{\occons}{\mathbin{::}}

\newcommand{\statecps}[3]{\llbracket #3 \rrbracket^{#2}_{#1}}
\newcommand{\cps}[2]{\llbracket #2 \rrbracket^{#1}}

\newcommand{\sem}[1]{\llbracket #1 \rrbracket}
\newcommand{\intens}[1]{\overline{#1}}

\newcommand{\keyword}[1]{\texttt{#1}}
\newcommand{\effect}[1]{\textbf{#1}}
\newcommand{\semdom}[1]{\textbf{#1}}
\newcommand{\handle}[2]{\keyword{with}\ #1\ \keyword{handle}\ #2}

\def\limp {\mathbin{{-}\mkern-3.5mu{\circ}}}

\title{Algebraic Effects and Handlers \\ in Natural Language Interpretation}
\author{Jiří Maršík \and Maxime Amblard}

\institute{LORIA, UMR 7503, Université de Lorraine, CNRS, Inria, Campus Scientifique, \\
F-54506 Vand\oe uvre-lès-Nancy, France \\
\email{\{jiri.marsik, maxime.amblard\}@loria.fr}}

\titlerunning{Effects and Handlers in Natural Language}
\authorrunning{Maršík and Amblard}
\pagestyle{empty}

\begin{document}

\maketitle

\begin{abstract}
  Phenomena on the syntax-semantics interface of natural languages have been
  observed to have links with programming language semantics, namely
  computational effects and evaluation order. We explore this connection to be
  able to profit from recent development in the study of effects. We propose
  adopting algebraic effects and handlers as tools for facilitating a uniform
  and integrated treatment of different non-compositional phenomena on the
  syntax-semantics interface. \par
  In this paper, we give an exposition of the framework of algebraic effects
  and handlers with an eye towards its applicability in computational
  semantics. We then present some exemplary analyses in the framework: we
  study the interplay of anaphora and quantification by translating the
  continuation-based dynamic logic of de Groote into a more DRT-like theory
  and we propose a treatment of overt wh-movement which avoids higher-order
  types in the syntax. \par
\end{abstract}

\section{Introduction}
\label{sec:motivation}

In the formal study of the syntax-semantics interface, researchers try to
discover a systematic translation from the syntactic structures of utterances
to their denotations. This translation is often performed indirectly by
translating the syntactic structures into a meta-language of semantic
representations. We will be studying the challenges of translating syntactic
structures to formulas of Church's higher-order logic.

A systematic account of the syntax-semantics interface should be
compositional, i.e. the denotations of complex utterances should be functions
of the utterances' constituents and their manners of composition. In the style
of abstract categorial grammars, we take our syntactic structures to be
$\lambda$-terms and we demand that the translation from syntax to semantics be
a homomorphism. This means that we view the syntactic structure as a program
and our goal is to find suitable definitions/interpretations for the
constructs of the language this program is written in. To describe the process
of building up and gluing together the semantic representation, we will use a
meta-language also. This language is often the same as the language of the
semantic representations, i.e. $\lambda$-calculus, which makes the boundary
between the two quite blurry.

Work in this paradigm has focused on phenomena that seem to defy the widely
accepted principle of compositionality. We will consider several examples of
such work now by looking at the case of a transitive verb. We will focus on
in-situ quantification and anaphora. However, similar phenomena could also
have been demonstrated on treatments of implicit arguments
\citep{blom2012implicit} or events \citep{qian2011event}.

%% \subsection{In-situ quantification}

\begin{exe}
  \ex \label{ex:quantification} Mary read every book. \\
  $\forall x. \textbf{book}(x) \to \textbf{read}(\textbf{Mary}, x)$
\end{exe}

To handle in-situ quantification, the verb is traditionally assigned the
denotation\footnote{As in Church's Simple Type Theory, we use
  $\iota$ for the type of individuals and $o$ for the type of propositions.}:

\vspace{-4mm}

\begin{align*}
  \sem{\textsc{read}} & : ((\iota \to o) \to o) \to ((\iota \to o) \to o) \to
  o \\
  \sem{\textsc{read}} & = \lambda s o. s (\lambda x. o (\lambda
  y. \textbf{read}(x, y)))
\end{align*}

One way to look at this is to say that NPs denote generalized quantifiers
(type $(\iota \to o) \to o$) and that transitive verbs are relations on
generalized quantifiers. We can also think of this as introducing control
effects (continuations) into our glue language
\citep{barker2002continuations}. NPs can have non-local effects on the
construction of the semantic representation by taking scope over their
continuations. $\sem{\textsc{read}}$ is the result of (partially)
CPS-transforming the simple $\lambda s o. \textbf{read}(s, o)$.

%% \subsection{Anaphora}
%% \label{ssec:anaphora}

\begin{exe}
  \ex \label{ex:anaphora} Mary$_1$ read her$_1$ favorite book. \\
  $\textbf{read}(\textbf{Mary}, \textbf{favorite-book}(\textbf{Mary}))$
\end{exe}

In order to interpret \eqref{ex:anaphora}, we need to link the antecedent with
the anaphoric pronoun since the semantic representation of the latter is
dependent on that of the former. In accordance with dynamic semantics
\citep{kamp1993discourse}, we can analyze this by positing a store into which
discourse referents are introduced and from which they are later
retrieved. This boils down to extending our glue language with state. This is
the strategy employed in the continuation-based dynamic logic of
\citet{de2006towards}. The type of NP denotations becomes in turn more complex
to reflect the fact that NPs can access the current state and manipulate their
continuations:

\vspace{-1mm}

$$
\sem{np} = (\iota \to \gamma \to (\gamma \to o) \to o) \to \gamma \to (\gamma
\to o) \to o
$$

The first argument (type $\iota \to \gamma \to (\gamma \to o) \to o$)
corresponds to a continuation delimited by the containing tensed clause, the
second argument (type $\gamma$) is the context, where we find e.g. the
available discourse referents, and the third argument (type $\gamma \to o$)
corresponds to an open-ended discourse-wide continuation. $\gamma$ lets us
access anaphoric state while the two different continuations serve to enforce
DRT accessibility constraints \citep{kamp1993discourse} (e.g. the universal
quantifier in \emph{everyone} provides a referent for and takes scope over
only its containing clause while the existential quantifier in \emph{someone}
provides a referent for and scopes over even following clauses).

The denotation this theory assigns to the transitive verb
is:\footnote{$\overline{o}$, a dynamic proposition, is shorthand for $\gamma
  \to (\gamma \to o) \to o$, where $\gamma$ is the anaphoric state, i.e. the
  referent store, and $\gamma \to o$ is the continuation.}

\begin{align*}
  \sem{\textsc{read}} & : \sem{np} \to \sem{np} \to \sem{s} \\
  \sem{\textsc{read}} & : ((\iota \to \overline{o}) \to \overline{o}) \to
                          ((\iota \to \overline{o}) \to \overline{o}) \to
                          \overline{o} \\
  \sem{\textsc{read}} & : ((\iota \to \gamma \to (\gamma \to o) \to o)
                           \to \gamma \to (\gamma \to o) \to o) \\ & \to
                          ((\iota \to \gamma \to (\gamma \to o) \to o)
                           \to \gamma \to (\gamma \to o) \to o) \\ & \to
                           \gamma \to (\gamma \to o) \to o \\
  \sem{\textsc{read}} & = \lambda s o. s (\lambda x. o (\lambda y e
  \phi. \textbf{read}(x, y) \land \phi e))
\end{align*}

We interpret the theory several ways. One can say that \textsc{read} is a
dynamic relation ($\alpha \to \beta \to \overline{o}$) on generalized dynamic
quantifiers (type $(\iota \to \overline{o}) \to \overline{o}$). We also hold
another view: instead of considering the dynamic proposition $\overline{o}$ as
a semantic representation, we see it as an effectful computation\footnote{In a
  sense similar to the notions of computation of \citet{moggi1991notions},
  popularized as the monads of functional programming.} that accesses some
anaphoric state and manipulates continuations to produce a term of type $o$,
the final semantic representation.


\section{Computation on the Syntax-Semantics Frontier}

In the preceding examples, we have seen that seemingly non-compositional
phenomena can be accounted for by admitting some sort of effect into our glue
language. If we go back to the metaphor of our syntactic structures being
programs that evaluate to their semantic representations, it seems that the
language these programs are written in exhibits a lot of effects. In the
previous chapters, we have seen state and continuations, but a look at
implicit arguments and events would lead us to also consider partiality and
environment-dependence, respectively.

Our chief motivation is to unite the treatments presented in the previous
section and to start formalizing the interactions between the phenomena
treated therein. Since effects are an intensely-studied topic in formal
semantics of programming languages, it would be convenient to employ a glue
language equipped with some notion of effects.

Choosing a language with some fixed set of side effects (such as some
general-purpose programming language) would not be practical since as we have
seen, treatments of newly discovered phenomena often call on new effects. We
would therefore prefer a framework that allows us to abstract over the
different effects in our glue language.

One such framework are the notions of computation of
\citet{moggi1991notions}, rendered as monads in category theory and used
heavily in popular functional programming. \citet{shan2002monads} already
examined the potential of a monadic framework for studying the various effects
he observed in existing linguistic analyses. However, in his study, he points
out that combining different monads is a difficult problem which still lacks a
satisfying solution (see references in \cite{shan2002monads} and
\citet{kammar2013handlers}).

A compelling alternative to the use of monads for describing effects has
gained popularity recently (\citet{bauer2012programming},
\citet{kammar2013handlers}, \citet{kiselyov2013extensible}). This new approach
is rooted in the algebraic study of effects and handlers by
\citet{plotkin2009handlers} that builds on the notion of Lawvere theory, which
provides a category-theoretical take on universal algebra different from the
one offered by monads \citep{hyland2007category}. We argue that algebraic
effects can solve some of our problems in combining multiple effects in a
single glue language and that the dual notion, handlers, can be identified in
existing linguistic treatments and is a natural fit for our task.


\section{The Case for Effects and Handlers}

In the algebraic effects framework, we use a calculus with abstract effectful
operations as language primitives \citep{kammar2013handlers}.

\begin{prooftree}
  \AxiomC{$(\effect{op} : A \to B) \in E$}
  \AxiomC{$\Gamma \vdash V : A$}
  \AxiomC{$\Gamma, x : B \vdash_E M : C$}
  \TrinaryInfC{$\Gamma \vdash_E \effect{op}\ V\ (\lambda x. M) : C$}
\end{prooftree}

We take $op$ to be an abstract operation with argument type $A$ and result
type $B$. We can invoke $op$ by providing it with an argument of the correct
type and a continuation which will receive the result of the operation and
carry out the rest of the computation.

What is of note in the above judgments is the $E$ subscript. Typing judgments
for computations not only serve to prove that a computation will finish by
yielding a value of some type (given after the colon), but they also guarantee
us that the computation will restrict itself to effectful operations contained
in the set $E$.

Such a type system should be familiar to anyone who has ever used Java. As
goes in the Java language specification \citep{gosling2000java}, ``Methods
declare the checked exceptions that can arise from their execution, which
allows compile-time checking to ensure that exceptional conditions are
handled''. We can see the invocation of an operation as throwing a (checked)
exception whose message contains the argument value and the continuation to be
called with the result.

The analogy of operations as exceptions will help us understand the dual
notion of a handler. Handlers behave exactly like exception
handlers. Exception handlers intercept specific types of exceptions to define
how they should be resolved, making use of any content stored in the exception
message. General handlers intercept specific abstract operations to define how
they should be executed, making use of the supplied argument and the callback
continuation. Since general handlers have access to the continuation, they can
resume the computation at the point where the abstract operation was invoked,
whereas exception handlers generally do not and thus discard the failed
computation.\footnote{The Common Lisp condition and restart system also allows
  one to resume the computation when handling an exception. However, the
  continuation available to the handler is not a first-class function and
  cannot be invoked multiple times.}

\subsection{Denotations for Effects}
\label{ssec:denotations-for-effects}

Let us now look at what would be a suitable denotational semantics for a
language with algebraic effects and handlers.

An expression $\Gamma \vdash_E M : C$ can be seen as something that will
either compute directly to the final value of type $C$ or something that will
invoke one of the operations in $E$, providing the operation's argument and a
continuation.\footnote{The idea of seeing a computation as denoting either a
  value or an effect dates back to at least \citet{cartwright1994extensible}.}
Following is a formula for the type of denotation given to a computation of type
$C$ exhibiting effects $E$. \citep{bauer2012programming}
\citep{kiselyov2013extensible}

$$
\sem{C_E} = C + \sum_{(\effect{op} : A \to B) \in E} A \times \sem{C_E}^B
$$

We can start appreciating some enticing features of this system in comparison
to the prevalent monadic approach.

First of all, the types of computations are decomposed into two components:
the type of the value being computed and the set of effects the computation
can have. This is in contrast with the monadic approach, where both the result
type and the effects meet in a single type, with the monad functor being
applied to the result type.

As an example, let us take a computation of result type $\alpha$ that accesses
state of type $\gamma$. In the effects framework, this computation would have
type $\alpha_{\left\{get : 1 \to \gamma, put : \gamma \to 1\right\}}$ where
$get$ and $put$ are the abstract operations used to read from and write to the
state. In the monadic framework, this computation would be rendered as
$State_\gamma(\alpha)$ which is equal to $\gamma \to \alpha \times \gamma$.

If we then consider extending this computation to include, e.g., exceptions,
we would get the following developments. In the effects framework, we have
$\alpha_{\left\{get : 1 \to \gamma, put : \gamma \to 1, raise : \epsilon \to
  0\right\}}$. In the monadic framework, we have
$Exc_\epsilon(State_\gamma(\alpha))$, which is $(\gamma \to \alpha \times
\gamma) + \epsilon$, or maybe $State_\gamma(Exc_\epsilon(\alpha))$, which is
$\gamma \to (\alpha + \epsilon) \times \gamma$.

If we look at the type of the denotation of the computation in the effects
framework, we see that the different effects sit side-by-side in an unordered
flat collection. The computation is represented as either a value or a request
to perform one of the effectful operations. Any interactions between these
effects are up to the handlers which will interpret this computation.

On the other hand, in the case of the monadic framework, we are forced to
commit from the start to a specific interaction between the two effects. The
representation of the computation is then bound to this specific
interpretation. Furthermore, the two modes of combining the two monadic
effects shown above are not exhaustive. \citet{kiselyov2013extensible} show an
example computation which combines exceptions and non-determinism and that
requires the use of the type $Exc_\epsilon(List(Exc_\epsilon(\alpha)))$.

The structure of the denotations in the effects framework makes it easy to
embed less expressive computations into contexts that employ a wider palette
of effects and to be polymorphic w.r.t. the effects. If we take the example of
a stateful computation in a context that also permits exceptions, then we can
benefit from the fact that $\sem{\alpha_{\left\{get, put\right\}}}$ is a
subset of $\sem{\alpha_{\left\{get, put, raise\right\}}}$ (modulo some trivial
injection). This means we can take expressions and denotations from linguistic
treatments that ignore some linguistic effect and we can plug them directly
into our richer integrated treatment without having to do any adaptation,
lifting or conversion.

More interestingly, we can also do the converse thanks to effect
polymorphism. We can replace any subterm $N_1$ of some term $M$ with another
subterm $N_2$ which has the same type but that also triggers some new
unhandled effects $E$. The resulting term $M'$ will be still well-typed and
the unhandled effects $E$ of $N_2$ will simply propagate to become the
unhandled effects of $M'$. This means that if we have a constituent below
which we need to introduce some effect and above which the effect will be
handled, we are not obliged to modify the denotation of the intervening
constituent. If it has no interaction with the effect, it can remain
agnostic. This is in contrast with the manual style of passing around all the
contextual arguments, states and continuations.

\subsection{Handlers}

Now we examine the notion of a handler. If we look back to the definition of
$\sem{C_E}$ in \ref{ssec:denotations-for-effects}, we see that a computation
is either a value or an effect. We can think of values as constants and of
effects as algebraic operations. More precisely, an effect $op : A \to B$ can
be seen as a family of algebraic operations $op_a$, one for each value $a$ of
type $A$. Each of these operations $op_a$ is a $\left| B \right|$-ary
operation on computations, combining the computations for all the possible
outcomes of the effect into a single computation.

To demonstrate on an example, let us consider an operation for writing strings
to some output channel, $print : \sigma \to 1$. This can be thought of as a
family of unary operations $print_s$ for every string $s$. The meaning of
$print_{hello}(C)$ is then a computation that first prints $hello$ to the
output channel and then carries out the computation $C$. The operator $or : 1
\to 2$ for non-deterministic choice corresponds to a binary operation on
computations such that $or(C_1, C_2)$ is a non-deterministic computation that
continues by either evaluating $C_1$ or $C_2$. This formulation in terms of
algebraic operations on computations is what gave this framework the name
``algebraic effects''.

In this algebraic view, a handler can be seen as a homomorphism which maps the
computation it handles to another computation by instantiating certain
abstract operations within \citep{plotkin2009handlers}
\citep{pretnar2010logic}.

Thinking of handlers as scoped interpretations is instructive. Denotations can
be written using environment-dependent abstract operations such as ``introduce
a new discourse referent (in the current DRS)'', ``access available discourse
referents'', ``access event under discussion'', ``access salient world'',
``scope over the current tensed clause''\ldots

Handlers can then be placed at the appropriate junctions in the syntactic tree
to give meaning to these abstract operations. To give several examples:
lexical items that necessitate a fresh DRS (negation, universal
quantification) will carry handlers for introducing and accessing discourse
referents; tensed clauses will use handlers to enforce quantifier scope
islands; access to the current event might be handled at the edge of a scope
domain; and modality operators would handle references to the salient world.

\section{Applications of Effects and Handlers to Computational Semantics}

Our first inspiration for adopting effects was to bridge the hierarchical
structures of DRSs and the compositional treatment of anaphora in the
continuation-based dynamic logic of \citet{de2006towards}.

\subsection{DRT, Continuations and Effects}

In \citet{de2006towards}, and in more detail in
\citet{lebedeva2012expression}, we can see indefinites introducing some
existential quantifier and adding the variable bound by the quantifier into
the store as a new discourse referent. In our effects framework, we can think
of this complex action as an effect, let us call it $\effect{fresh}$, whose
purpose is to introduce a new discourse referent at the current point of
discourse and whose implementation will introduce a quantifier someplace and
record the referent in some store. This decomposition into an abstract
interface (the $\effect{fresh}$ operation which introduces a discourse
referent) and specific implementations that will realize it (handlers for
$\effect{fresh}$) is pertinent. It gives us another way to explain how it is
possible that the same indefinite expression can sometimes contribute an
existential quantifier and sometimes a universal one, as is the case with the
famous example of donkey sentences \citep{kamp1993discourse}.

Besides introducing discourse referents, lexical items in continuation-based
dynamic logic also access the sum total of the available discourse referents
to be able to resolve anaphora by choosing a salient antecedent from their
ranks. We model this feature by an effect, $\effect{get}$, which lets
computations access the current contents of the discourse store.

With these two operations in hand, we can start rewriting de Groote's
continuation-based dynamic logic. In short: the dynamic existential quantifier
will essentially boil down to a call to $\effect{fresh}$, dynamic negation
will introduce a handler (a new DRS) and dynamic conjunction will be simple
conjunction (with arguments evaluated left-to-right).

\vspace{-4mm}

\begin{align*}
  \dexists &= \lambda P. P (\effect{fresh}\ ()) \\
  \dnot &= \lambda P_t. \{ \lnot (\handle{drs\ (\effect{get}\ ())}{P_t!}) \} \\
  \dand &= \lambda P_t Q_t. \{ P_t! \land Q_t! \}
\end{align*}

The fragments we show are faithful excerpts from our experimental grammar. The
language used is \emph{Eff} \citep{bauer2012programming}. As we are dealing
with effects, the order of evaluation matters and since \emph{Eff} is
call-by-value, we use thunks to simulate call-by-name. For those, we adopt the
notation used in \citet {kammar2013handlers}. $\{ M \}$ is short for $\lambda
(). M$ (i.e. a dummy abstraction serving to build a thunk) and $M!$ is short
for $M\ ()$ (i.e. applying a thunk to the dummy unit argument). Furthermore,
we mark variables that hold thunks with the subscript $t$. The exposition
could be cleared up by devising a language which separates strictness from
abstraction, in much the same way as was already proposed by
\citet{kiselyov2008call}.

We can now move the dynamicity of continuation-based dynamic logic into the
effects, and so instead of having the type $(\iota \to \gamma \to (\gamma \to
o) \to o) \to \gamma \to (\gamma \to o) \to o$ for NP denotations, we get
$(\iota \to o_E) \to o_E$ for some set of effects $E$ containing
$\effect{fresh}$ and $\effect{get}$.

\vspace{-4mm}

\begin{align*}
  \sem{\textsc{she}} &= \lambda k. k (sel_{she}' (\effect{get}\ ())) \\
  \sem{\textsc{something}} &= \lambda k. \dexists x. (k\ x) = \lambda k. k
  (\effect{fresh}\ ()) \\
  \sem{\textsc{every}} &= \lambda n k. \dforall x. (\{n\ x\} \dimpl \{k\ x\})! \\
  \sem{\textsc{read}} &= \lambda S O. S (\lambda s. O (\lambda
  o. \semdom{read}(s, o)))
\end{align*}

We have switched from dynamic propositions ($\overline{o} = \gamma \to (\gamma
\to o) \to o$) to effectful computations of static propositions ($o_E$), as
was hinted at in Section~\ref{sec:motivation}.

The NP denotations now have type $(\iota \to o_E) \to o_E$. This is still more
complicated than the naive type $\iota$ one starts with for proper nouns. We
can see this type as a computation of $\iota$ with access to some continuation
of result type $o_E$. If we look at the denotations, this continuation (the
argument $k$) is properly exploited only in the denotation of \emph{every}
where it serves to mark the scope of the quantifier. This scope is supplied to
the NP by the tensed clause it is contained in. If we were to take our
approach to its logical conclusion, we could turn this into an effect as
well. Quantificational noun phrases could employ a $\effect{scope\_over}$
effect to introduce some quantifier and tensed clauses could be handlers for
these effects, delimiting the scope of these quantifiers.

\vspace{-4mm}

\begin{align*}
  \sem{\textsc{she}} &= \{ sel_{she}' (\effect{get}\ ()) \} \\
  \sem{\textsc{something}} &= \{ \effect{fresh}\ () \} \\
  \sem{\textsc{every}} &= \lambda n. \{ \effect{scope\_over}\ (\lambda
  k. \dforall x. (\{n\ x\} \dimpl \{k\ x\})!) \} \\
  \sem{\textsc{read}} &= \lambda s_t o_t. \{ \handle{tensed\_clause}{\semdom{read}(s_t!, o_t!)} \}
\end{align*}

We have now arrived at a type for NP denotations that is $\iota_E$, where $E$
is some set of effects including $\effect{scope\_over}$, $\effect{get}$ and
$\effect{fresh}$, the three effects being somewhat analogous to the three
arguments that NP denotations abstract over in continuation-based dynamic
logic. For now, we have restricted our set of effects that NPs can use to
modify the current DRS to just $\effect{fresh}$, which lets us only introduce
new referents. However, indefinite expressions such as \emph{a good book} not
only introduce referents but also assert something about them. We can model
this by admitting a new effect, $\effect{assert}$, that will be handled by the
same handler as $\effect{fresh}$ and $\effect{get}$, i.e. by a DRS.

\vspace{-4mm}

\begin{align*}
  \sem{\textsc{some}} = \lambda n. \{\ 
    &\keyword{let}\ x = \effect{fresh}\ ()\ \keyword{in} \\
    &\keyword{let}\ () = \effect{assert}\ (n\ x)\ \keyword{in} \\
    &x \ \}
\end{align*}

We have arrived at the following system with two kinds of handlers. We have
DRSs as handlers that handle three effects that can be seen as the interface
of a DRS: we can build up a DRS by introducing new discourse referents
($\effect{fresh}$) and constraints ($\effect{assert}$) and we can use the
information stored inside to resolve anaphora ($\effect{get}$). We also have
tensed clauses as handlers. They delimit scope islands which define the scopes
of the quantifiers contained within ($\effect{scope\_over}$).

\subsection{Overt Movement and Effects}

Overt movement (as in the case of an object extracted from a relative clause
by a relativizer such as \emph{who}) has in the categorial school been often
studied as $\lambda$-abstraction or hypothetical reasoning.

\vspace{-3mm}

$$
\textsc{who} : (np \limp s) \limp n \limp n
$$

Using linear implication for the arrow type lets us enforce that one
relativizer can only cover one trace. However, overt wh-movement has more
restrictions, especially when it comes to multiple extraction
\citep{pogodalla2012controlling}.

When we have effects available to us, we can choose an alternative
solution. Traces can be seen as inaudible NPs whose semantic representation is
computed through an effectful operation, $\effect{move}$. A relativizer will
then handle $\effect{move}$ events in the relative clause. The handler will be
a shallow handler \citep{kammar2013handlers}, meaning it will only handle the
first occurrence of the $\effect{move}$ event that will occur in the relative
clause, which gives correct predictions in the case of multiple extraction.

\vspace{-4mm}

\begin{align*}
  \sem{\textsc{who}} = \lambda r_t n. \lambda x.\ & \keyword{let}\ r =
  \handle{extract}{r_t!}\ \keyword{in} \\
   & (n\ x) \land (r\ x) \\
\end{align*}

\vspace{-8mm}

\hspace{2.5cm} $ \sem{\epsilon} = \{ \effect{move}\ () \} $ \\

By making the relativizer handler fail in the case that it does not intercept
any $\effect{move}$ effect (i.e. the relative clause contained no free trace),
sentences with ungrammatical uses of the relativizer can be ruled out. With
this technique, the type of \textsc{who} becomes $s \limp n \limp n$, which is
interesting because using 2nd order types, such as this one, in the syntax
makes it possible to use efficient parsing algorithms
\citep{kanazawa2007parsing}.


\section{Conclusion}

In this paper, we gave a short exposition to the theory of algebraic effects
and handlers and demonstrated how it can be applied to computational
semantics. Besides our principal example of dynamic logic and DRT, we also
covered extraction. However, that is not the full extent of the applicability
of effects and handlers. The event argument under discussion
\citep{qian2011event} and the salient world are both examples of dynamic
binding, which is a special case of a handler. \citet{lebedeva2012expression}
proposes modeling presuppositions and proper names using exceptions, which are
a trivial case of handlers. The treatment of implicit arguments in
\citet{blom2012implicit} uses an operator which is also a simple exception
handler.

We have not spent much time defending the position that algebraic effects and
handlers facilitate the analysis of \emph{multiple} phenomena in the same
grammar. However, compelling evidence for using algebraic effects and handlers
in contexts that involve multiple computational effects at the same time can
be found in existing literature (e.g. \citet{kiselyov2013extensible},
\citet{cartwright1994extensible}).

For future work, we are interested in developing a simple calculus with
effects and a notion of evaluation order suitable for the study of the
syntax-semantics interface. Having fixed such a calculus, we would aim to
study non-compositional semantic phenomena in concert, not in isolation, and
to examine their interactions.

% CITE! CBN, Shan

\bibliography{effects-paper}
\bibliographystyle{plainnat}

\end{document}
